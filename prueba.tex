\documentclass{article}
\usepackage[utf8]{inputenc}
\usepackage{csquotes}
\usepackage[spanish]{babel}
\usepackage{biblatex}
\usepackage[usenames]{color}
\definecolor{blue}{RGB}{14,139,125}
\begin{document}
\section*{Estática de fluidos pre-lectura 1}
\paragraph*{}
La materia se suele dividir en sólidos y fluidos \textcolor{blue}{(donde se incluyen a los gases y los líquidos)},
estos últimos definiéndose como un conjunto de moléculas débilmente unidas las cuales están dispersas, teniendo una propiedad peculiar la cual es que a diferencia
de un solido en el cual se puede aplicar una fuerza en cualquier angulo respecto a su superficie soportando fuerzas de corte y tension, y al fluido no soportar estas fuerzas 
al aplicarle cualquier fuerza esta siempre se propagara perpendicularmente a la superficie de contacto, a esta fuerza
se le llama presión y se define como la fuerza aplicada entre el area de contacto: $P =\frac{F}{A}$, siendo P un valor
escalar ya que según establece el principio de pascal cualquier presión ejercida por un liquido se esparce uniformemente por este no teniendo una dirección privilegiada.\\
algo curioso de la presión es que no importa la forma del recipiente o la cantidad de liquido esta no aumenta conforme a ello,
sino que aumenta según la altura o profundidad que estés sumergido en el liquido
para demostrarlo  un ejemplo muy bueno es:
si tenemos un cilindro de area transversal $A$ que es el area en contacto con el liquido y altura $h$ y lo llenamos de un liquido de densidad $\rho$ contante
y después lo sumergimos una distancia $d$ en el mismo liquido al hacer el análisis de fuerzas sobre el objeto queda que
$\Sigma \vec{F}=PA-{P}_{0}A-Mg=0$, siendo $PA$ la presión ejercida en la parte inferior del cilindro, ${P}_{0}A$ la presión ejercida encima y Mg el peso del cilindro
con esto sabido podemos definir a la masa como la densidad del por el volumen $M=\rho V$
y siendo V el volumen del cilindro que seria el area transversal por la altura, $V=Ah$
quedando que $M=\rho Ah$ sustituyendo queda $PA-{P}_{0}A-\rho hA$ al final al despejar y simplificar queda
$P={P}_{0}+\rho hg$ siendo ${P}_{0}$ $g$ y $\rho$ constantes y $h$ la única variable
siendo $\rho$ la densidad que depende del liquido a tratar, $g$ la aceleración de la gravedad en la tierra y
${P}_{0}$ la presión inicial que en el caso de la tierra se refiere siempre a la presión atmosférica, la cual se origina
gracias que que la atmósfera también es un fluido por lo cual ejerce presión sobre nosotros y sobre cualquier cosa debajo de ella siendo asi entonces la presión
que se mide al restar la presión atmosférica aquella que seria la que ejercería el liquido se conoce como presión manométrica.
en el ejemplo anterior para demostrar que la presión depende de la altura tal como esta definido se asume
que este queda suspendido, ya que la fuerza que cancela la gravedad es la presión ejercida de abajo hacia arriba, por tercera ley de newton y la definición de presión
se puede suponer que esto se debe a que el peso ejerce una fuerza sobre el liquido y este responde con una fuerza igual incluso mayor ya que también tiene que cancelar la fuerza de la presión inicial
entonces de donde viene esa fuerza extra,
una pista de lo que puede estar pasando es  algo que no menciona el problema
es que en la vida real ademas de ver al cilindro quedar suspendido también veremos que al momento de sumergirlo parecerá que el nivel del agua aumenta
y esto se debe a que el cilindro desplaza una cantidad de agua que ejerce una fuerza extra de empuje, esa fuerza extra es la ayudaba a mantener el objeto suspendido y se trata de la fuerza de flotación
el mecanismo por el cual funciona esta fuerza fue descubierto por el matemático, físico e ingeniero griego Arquímedes de Siracusa siendo este mecanismo llamado en su honor
el principio de arquimedes el cual va de la siguiente manera ´´todo objeto sumergido en un fluido sufre un empuje igual al del peso del fluido que desplaza'';
esta fuerza depende de la densidad del objeto sumergido pasando lo siguiente\\
\begin{enumerate}
    \item si la densidad del objeto es menor a la del liquido flotara
    \item si la densidad es igual este quedara suspendido en el liquido
    \item si la densidad es menor este se hundirá 
\end{enumerate}
\end{document} 