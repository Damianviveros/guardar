\documentclass{article}
\usepackage[spanish]{babel}
\usepackage[utf8]{inputenc}
\begin{document}
\section*{Máquinas térmicas y la Maquina de Carnot}
cuando hablamos de maquinas térmicas nos referimos a un aparato o maquina la cual mediante un proceso termodinámico cíclico (que vuelve al mismo punto), 
convierte parte de la energía absorbida en en forma de calor en un trabajo, el ejemplo mas clásico serian las maquinas de vapor que mediante la quema de un
combustible como lo es el carbon, genera calor que es absorbido por el agua cambiando de fase y por consiguiente expandiéndose y usando esa expansion para
generar trabajo, la idea de maquina térmica es sencilla, tienes una sustancia en la cual aplicas pones entre dos reservorios o baños térmicos, uno 
teniendo una temperatura $t_h$ y otro una temperatura $t_c$ donde $t_h>t_c$, entonces al poner esta maquina entre estos dos repositorios lo que pasara 
es que esta absorberá calor del repositorio con mayor temperatura, la maquina hará algún proceso termodinámico cíclico que al finalizar generara trabajo 
que saldrá afuera del sistema y el calor que no pudo ser convertido en trabajo terminara siendo depositado en el reservorio con la temperatura menor,
ahora esto es importante ya que existe algo que se llama eficiencia de la maquina que nos es mas que una division entre el trabajo generado y el calor absorbido
y de esta definición nace la formulación Kelvin-Planck de la segunda ley de la termodinámica que dice ``no existe maquina térmica que realizando un proceso
termodinámico cíclico convierta todo el calor absorbido en trabajo'', pero si existe una maquina que al absorber calor lo convierta en trabajo no existirá
una maquina que al recibir trabajo transfiera calor, la respuesta es si y se llama bomba térmica o mejor conocida como refrigerador, que básicamente se
trata de una maquina que al estar conectada a los mismos reservorios pero en vez de absorber calor del reservorio con mayor temperatura absorbe del de menor 
temperatura y lo transfiere al de mayor al realizar un trabajo sobre la maquina y esto nos lleva al segundo enunciado de la ley de la termodinámica 
formulado por Clausius que dice ``no existe maquina térmica que transfiera de manera continua calor de un objeto de menor temperatura a otro con mayor 
temperatura sin que se realize un trabajo sobre ella''
\end{document}