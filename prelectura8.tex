\documentclass{article}
\usepackage[spanish]{babel}
\usepackage[utf8]{inputenc}
\begin{document}
\section*{Máquinas térmicas y la Maquina de Carnot}
cuando hablamos de maquinas térmicas nos referimos a un aparato o maquina la cual mediante un proceso termodinámico cíclico (que vuelve al mismo punto), 
convierte parte de la energía absorbida en en forma de calor en un trabajo, el ejemplo mas clásico serian las maquinas de vapor que mediante la quema de un
combustible como lo es el carbon, genera calor que es absorbido por el agua cambiando de fase y por consiguiente expandiéndose y usando esa expansion para
generar trabajo, la idea de maquina térmica es sencilla, tienes una sustancia en la cual aplicas pones entre dos reservorios o baños térmicos, uno 
teniendo una temperatura $t_h$ y otro una temperatura $t_c$ donde $t_h>t_c$, entonces al poner esta maquina entre estos dos repositorios lo que pasara 
es que esta absorberá calor del repositorio con mayor temperatura, la maquina hará algún proceso termodinámico cíclico que al finalizar generara trabajo 
que saldrá afuera del sistema y el calor que no pudo ser convertido en trabajo terminara siendo depositado en el reservorio con la temperatura menor,
ahora esto es importante ya que existe algo que se llama eficiencia de la maquina que nos es mas que una division entre el trabajo generado y el calor absorbido
y de esta definición nace la formulación Kelvin-Planck de la segunda ley de la termodinámica que dice ``no existe maquina térmica que realizando un proceso
termodinámico cíclico convierta todo el calor absorbido en trabajo'', pero si existe una maquina que al absorber calor lo convierta en trabajo no existirá
una maquina que al recibir trabajo transfiera calor, la respuesta es si y se llama bomba térmica o mejor conocida como refrigerador, que básicamente se
trata de una maquina que al estar conectada a los mismos reservorios pero en vez de absorber calor del reservorio con mayor temperatura absorbe del de menor 
temperatura y lo transfiere al de mayor al realizar un trabajo sobre la maquina y esto nos lleva al segundo enunciado de la ley de la termodinámica 
formulado por Clausius que dice ``no existe maquina térmica que transfiera de manera continua calor de un objeto de menor temperatura a otro con mayor 
temperatura sin que se realize un trabajo sobre ella'', de estos conceptos se desprende otro muy importante que son los procesos reversibles e irreversibles
necesarios para entender la eficiencia y en resumidas cuentas todo proceso reversible sera aquel que pueda volver al estado inicial siguiendo el mismo
camino por el caul cambio de estado, y un proceso irreversible es todo aquel que no sea reversible, cabe destacar que todos los procesos naturales
o procesos instantáneos son irreversibles ya que por como funciona el ambiente la energía de estos siempre termina disipándose en forma de fricción o calor
por lo tanto nunca se puede replicar el proceso inverso sin agregar trabajo o energía al sistema por lo cual este no es reversible.
Bueno esto nos sirve para introducir el concepto de maquina de Carnot la cual es una maquina teórica que tiene la mayor eficiencia posible,
esta funciona a traves de un ciclo termodinámico ideal mejor conocido como ciclo de Carnot lo interesante es que Carnot demostró que ninguna maquina 
térmica puede superar la eficiencia de la maquina de Carnot enunciando el que se conoce como teorema de Carnot que dice ``ninguna maquina térmica que funcione
entre dos depósitos de energía puede ser mas eficiente que una maquina de Carnot que este entre esos mismos depósitos''
la comprobación es sencilla básicamente si tienes dos maquinas conectadas entre si tal que una de ellas ejerce trabajo sobre la otra y una de ellas
tiene una eficiencia igual a la de la maquina de Carnot y la otra tiene una eficiencia mayor sobre la primera veras si las conectas ambas a los mismos
baños térmicos y las pones a trabajar en conjunto las dos maquinas funcionaran como una bomba de calor y dado a que el trabajo que se ejerce de manera
externa es igual a 0 (el trabajo que ejerce una sobre otra es interno) esto violaría la segunda ley de la termodinámica de acuerdo con el enunciado de Clausius
por lo tanto no puede existir una maquina que supere la eficiencia de la maquina de Carnot y dado que en el mundo real el mismo principio de la maquina de Carnot
es casi imposible debido a las perdidas de energía por la fricción entre las superficies una maquina real nunca podrá superar la eficiencia de las maquina de Carnot
\end{document}