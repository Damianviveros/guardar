\documentclass{article}
\usepackage[spanish]{babel}
\usepackage[utf8]{inputenc}
\begin{document}
\section*{Modos de transferencia de energía térmica}
\paragraph*{}
cuando hablamos de transferencia de energía térmica nos referimos a como el calor se mueve
de un lugar a otro, en el tema de trabajo vimos como la variación de la energía interna de un
sistema era igual al trabajo que se ejerce sobre el sistema mas el calor trasferido hacia el 
sistema,esencial mente el calor se transfiere de tres maneras distintas.
la conducción térmica, la manera mas común de transferencia de calor, esta se debe a la interacción 
entre constituyentes (átomos,moléculas,partículas), se interpreta como el intercambio de energía cinética
entre constituyentes donde las partículas con mas energía cinética transfieren parte de su energía a las
de menos energía mediante colisiones, un ejemplo de conducción térmica seria cuando tomas un pedazo de metal
caliente entre tus manos al hacerlo notaremos como nuestra mano empieza a calentarse ya que la barra transfiere
calor a nuestra mano mediante conducción, o por ejemplo si pones un trozo de metal encima de la llama de una vela
veremos como poco a poco este se ira calentando y transferirá el calor a lo largo de su cuerpo hasta llegar a nuestra
mano esto mediante conducción térmica, pero si por ejemplo en vez de un trozo de metal pusiéramos un trozo de por ejemplo
concreto al fuego podríamos dejarlo durante mucho tiempo y este no pasaría nada, esto debido a que el concreto es pésimo 
conduciendo el calor a diferencia de muchos metales, esta capacidad de conducir el calor mejor o peor es capturada
por el coeficiente de conductividad térmica k que no es mas que un coeficiente de proporcionalidad entre la transferencia
de energía hacia un sistema y la variación de temperatura que este presenta.
el segundo método de transferencia es la convección térmica que podría definirse como la trasferencia de calor mediante
un medio como seria el agua o el aire, el ejemplo mas fácil es cuando pones las manos sobre la llama de una vela sin tocar 
la llama te das cuenta que si dejas tu mano encima de la llama esta se comienza a calentar, esto es debido básicamente al medio
que los rodea el cual es el aire, cuando el aire se calienta este sube ya que cambia su densidad, al subir envuelve tu mano
transfiriendo parte de su energía hacia tu mano y esto es a lo que se llama convección y siendo esto la forma en que los
radiadores calientan una habitación.
el ultimo mecanismo es la radiación,
cuando decimos radiación nos referimos a la radiación electromagnética emitida y absorbida por diferentes objetos,
como se sabe la temperatura no es mas que la energía cinética de sus átomos, esto implica que se mueven y como sabemos los
átomos están hechos de electrones y protones los cuales tienen carga y resulta que cuando un objeto con carga se mueve este
genera ondas electromagnéticas, resulta que cuando resulta que la rapidez con la que un cuerpo irradia estas ondas esta 
relacionada con la potencia cuarta de su temperatura y con un coeficiente de emisividad que va de 0 a 1 que es igual a la 
absortividad que es la fracción de radiación que absorbe un objeto algo curioso es que cuando un objeto esta en equilibrio
con su alrededor la cantidad de radiación que emite es la misma que absorbe, y en el caso de que el objeto este mas caliente 
que el medio pierde mas rápido energía de la que absorbe.

\end{document}