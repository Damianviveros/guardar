\documentclass{article}
\usepackage[utf8]{inputenc}
\usepackage[spanish]{babel}
\usepackage[usenames]{color}
\begin{document}
\section*{Movimiento de los fluidos }
\paragraph*{}
cuando un fluido se mueve se dice que este tiene un flujo, este flujo se puede clasificar es dos tipos, el laminar el cual es en el que todas las 
partículas del fluido se mueven a la misma velocidad y no chocan entre si generando este flujo, en cambio cuando estas llegan cierta velocidad conocida 
como velocidad critica este se convierte en un flujo turbulento el cual se caracteriza por los pequeños torbellinos que se generan en el,
ejemplo cuando apagas una vela y ves el humo que sale de ella el humo tiende a generar pequeños vortices debido a que el aire que nos rodea genera 
siempre flujos turbulentos, otra característica que tienen los fluidos relacionada con su movimiento en lo que se conoce como viscosidad, esta se trata de 
la resistencia del fluido a moverse o fluir siendo algo asi como la fricción de un fluido respecto a si mismo ya que esta característica
es propia del fluido y no de su entorno y por ultimo el movimiento que sigue una partícula en el fluido se llama corriente y la velocidad de esta partícula siempre es
tangencial a la corriente, para facilitarnos el estudio de los mismos tendremos ciertas concesiones como:\\
\begin{enumerate}
    \item el fluido es incompresible
    \item el fluido no tiene viscosidad
    \item el fluido es no rotatorio
    \item el flujo es estacionario
\end{enumerate}
teniendo esto en cuenta también hay otra característica que pasa cuando un fluido se mueve es que estos conservan su masa en todo momento;
Se deriva este resultado a partir de lo siguiente, en un tubo no uniforme tomamos un punto $\Delta {x}_{1}$ al hacer pasar un fluido por ese punto en un tiempo $\Delta t$
este recorre un  área ${A}_1$ inicial,al tomar en el mismo tubo en otra posición $\Delta {x}_{2}$ recorre un área A2 en el mismo intervalo de tiempo $\Delta t$
la masa esta dada por ${m}_{1}=\rho V$ y siendo volumen $V=\Delta {x}_{1} {A}_{1}$ tenemos que ${m}_{1}=\rho \Delta {x}_{1} {A}_{1}$ y ${m}_{2}=\rho \Delta {x}_{2} {A}_{2}$
y sabiendo que $\Delta {x}_{1} =  {v}_{1}\Delta t$y$\Delta {x}_{2} =  {v}_{2}\Delta t$
y al ser $\rho$ constante por ser incompresible el fluido y el intervalo $\Delta t$ por el que se mueven es igual y no hay entrada ni salida
de fluido entre ${x}_{1}$ y ${x}_{2}$ queda que ${m}_{1}={m}_{2}$
Usando esta relación obtenemos que
$\rho {A}_{1}{v}_{1}=\rho{A}_{2}{v}_{2}$ simplificando la densidad
obtenemos ${A}_{1}{v}_{1}={A}_{2}{v}_{2}$
siendo está la fórmula o ecuación de continuidad.\\
Ahora algo curioso que pasa si los fluidos se mueven es que al parecer su presión tiende a variar
un ejemplo es cuando soplas encima de una hoja y esta sube en vez de bajar,
este fenomeno se puede describir suponiendo que tienes un tubo no uniforme y en el
ejerces fuerza en los dos extremos del tubo desplazando fluido y por ende generando trabajo
como el fluido es incompresible entonces supones que los trabajos son iguales ${W}_{1}={W}_{2}$
ahora definimos el trabajo 1 como ${W}_{1}={F}_{1}{x}_{1}$ y el trabajo 2
${W}_{2}={F}_{2}{x}_{2}$ y con el resultado anterior sabemos que el volumen de ambos tramos
de fluido siempre es el mismo que del resto del fluido por que la masa siempre es las misma y la densidad igual
entonces para saber la fuerza que ejercen ambos tramos fluidos recurrimos a $P=\frac{F}{A}$
por lo cual ${W}_{1}={P}_{1}{A}_{1}{x}_{1}$ y ${W}_{2}=-{P}_{2}{A}_{}{x}_{2}$ siendo ${W}_{2}$ negativa porque 
es ejercida contrariamente a ${W}_{1}$ entonces definimos al trabajo total como
${W}_{t}={W}_{1}-{W}_{2}$
simplificando ${W}_{t}=({P}_{1}-{P}_{2})V$
y al usar teorema trabajo energía sumando la energía potencial de cada segmento dada
por su masa y la altura obtenemos
$({P}_{1}-{P}_{2})V=\frac{1}{2}m{v}_{1}^{2}-\frac{1}{2}{v}_{2}^{2}+mg{y}_{1}-mg{y}_{2}$ y al dividir todo entre volumen y simplificando queda
${P}_{1}+\rho {v}_{1}^{2}+\rho g{y}_{1}={P}_{2}+\rho {v}_{2}^{2}+\rho g{y}_{2}$
siendo esta la conocida ecuación de Bernoulli que nos dice que la presión de un fluido
varia de acuerdo a que tan rápido se mueve, a pesar de ser esta deducida a partir de un fluido
incompresible se a demostrado que funciona incluso con los gases, una nota interesante
es que como esta ley se obtiene a partir del teorema trabajo energía y sabemos que
el resultado de esta siempre es contante esto quiere decir que la energía en el fluido siempre
se conserva.
por ultimo en el principio hable de la viscosidad pero hay algo que la viscosidad 
determina de un flujo y es si este es turbulento, laminar o caótico a partir de un 
numero conocido como numero de Reynolds el cual determina que tipo de flujo sera
y para calcularlo se usa $Re=\frac{\sqrt{fuerzas cineticas}}{\sqrt{fuerzas viscosas}}$
y esto es importante ya que normalmente para medir la viscosidad de un fluido se usa 
la llamada ley de stokes que es un caso particular de las ecuaciones de navier-stokes
donde se calcula la fuerza de fricción que experimenta una bola sometida a un flujo laminar con un numero de Reynolds
bajo

\end{document}s