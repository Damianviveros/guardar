\documentclass{article}
\usepackage[spanish]{babel}
\usepackage[utf8]{inputenc}
\begin{document}
\section*{Trabajo y calor}
Como sabemos el trabajo que hace una fuerza no es mas que la integral de la de la fuerza en función de la distancia, entonces análogamente si pusiéramos un piston
sobre un gas y este gas suponiendo que es un gas en regimen ideal cuando aumentamos su temperatura vemos que este genera un trabajo pero este es negativo respecto gas
sucediendo contrariamente si por ejemplo quitáramos temperatura al gas, tal vez pareciendo algo sin sentido pero al verlo visualmente en una gráfica de volumen
contra presión nos daremos cuenta que pasar de un estado de $v_f$ al estado inicial el trabajo necesario es positivo y pasar de $v_i$ al estado de volumen final
este se torna negativo dándonos la pista de que la presión como tal es una fuerza no conservativa por lo cual este trabajo es negativo,
pero que pasa si la presión o el volumen son constates, incluso si la temperatura es constates, pues en los primeros dos es fácil ya que al ser la presión constate
esta sale de la integral por lo cual no afecta al trabajo mas que en un factor numero, pero si el volumen es constate esto significa que el trabajo realizado es cero;
cuando hablamos de temperatura constate nos referimos a un proceso isotérmico, por ende su gráfica se llama isoterma esta representa el trabajo que se hace al pasar
de un punto ``a'' de volumen y presión a un punto ``b'' el trabajo es también negativo esto obtenido a partir de la ecuación de un gas ideal; y por ultimo
en el caso del que el sistema este aislado termicamente a este proceso se le llama adiabatico en el cual el cambio también esta dado por la ecuación general de gases
ideales y como todo trabajo ejercido por la presión este suele ser negativo.
ahora como sabemos por el teorema trabajo energía el trabajo es igual al cambio de energía y como vimos anteriormente la energía interna de un gas depende unicamente
de la temperatura y no de su volumen o presión, entonces cuando realizamos cualquier cambio en la temperatura esta se expresa como un cambio de la energía del sistema,
lo cual implica un trabajo, esto es importante ya que según la ley cero de la termodinámica cuando pones en contacto térmico dos sistemas estos tienen a quedar en un 
estado de equilibrio con una temperatura igual, y como vimos un cambio de temperatura implica un cambio de energía y un trabajo entonces eso significa que 
lo que en realidad hacen es cambiar energía este intercambio de energía se le conoce como calor y por newton sabemos que al intercambiar energía el cuerpo con menos
energía recibe energía del cuerpo que mas energía tiene por lo tanto esto es una explicación del porque el calor va del cuerpo caliente al cuerpo frio ya que 
a nivel de energía el cuerpo caliente cuenta con mas energía que el cuerpo frio, pero algo que recalcar es como sabemos un cambio de energía implica un trabajo
esto significa que si encuentras la manera de aprovechar ese trabajo y convertirlo en movimiento obtienes lo que se conoce como motor térmico que como se ve mas
adelante son la base de las maquinas de Carnot, incluso viceversa si encuentras la manera de convertir esa energía del calor en otro tipo de energía, como por ejemplo
la energía eléctrica obtienes las bases de un generador eléctrico moderno, siendo este concepto de convertir el calor producido por poner objetos de diferentes 
temperaturas en contacto térmico en trabajo mecánico lo que llevo a la primera revolución industrial con las maquinas de vapor 
\end{document}