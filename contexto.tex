\documentclass{article}
\usepackage[utf8]{inputenc}
\usepackage{url}
\usepackage[spanish]{babel} 
\begin{document}
\section*{Contexto historico}
EL ciclo de Brayton, llamado así en honor al inventor e ingeniero estadounidense 
George Brayton,
aparece por primera vez en la patente del inventor John Barber en 1791 patente asociada
a una maquina de gas, la cual debido a su rudimentario sistema de compresión e 
ineficiente sistema de calentamiento llevo al estrepitoso fracaso del motor frente
a la maquina de vapor cayendo en el olvido.
Mas tarde en la década de 18840 el físico británico James Prescott Joule 
planteo un ciclo termodinámico para motores de aire en su obra de 1845
"On the Mechanical Equivalent of Heat" ciclo que es termodinámicamente equivalente al 
ciclo de Brayton, su trabajo se limito solo al ámbito teórico, al reconocer que 
la implementación de este requería o bien de elevados costes de combustible, o
sistemas de compresión de gas extremadamente grandes y resistentes.
En 1872 George Brayton presento la patente de su \textit{Ready Motor}.
En su patente, basada en un motor de pistones de flujo discontinuo, 
la compresión se realizaría en un cilindro, tras lo cual el aire comprimido, 
que habría pasado a una cámara de calentamiento, se calentaría por una fuente de 
calor externa, para finalmente expandirse en el cilindro de expansión, produciendo 
un trabajo, teniendo este graves problemas como Joule ya había anticipado a pesar de ello
en 1875 inventor y ingeniero irlandés utiliza un motor de Brayton para impulsar el primer 
submarino autopropulsado,
en 1875 el inventor estadounidense George.B Selden patenta el primer automóvil
de combustión interna basado en el ciclo de Brayton, mas tarde en 1887 el mismo Brayton
desarrolla y patenta un motor de aceite de inyección directa de cuatro tiempos mejorando
el ciclo aunque teniendo todavía varios problemas de implementación. A principios del
siglo 20 se empieza el desarrollo de la turbina de gas en consecuencia de solucionar
el principal problema tecnico asociado al ciclo de Brayton, la etapa de compresión de
gas ya que la compresión de un fluido compresible no es sencilla ya que 
requería de pistones que volvían a los motores grandes y pesados o bien de 
un compresor cuyo proceso de compresión requería de trasladar a un fluido de una zona 
de bajas presiones a otra de altas presiones, proceso el cual poco favorecido por la 
termodinámica siendo muy poco efectivo.
Con el avance en mecánica de fluidos y el descubrimiento de nuevos materiales 
en 1927 el ingeniero ingles Frank Whittle  aplica el ciclo a una
turbina de gas para propulsión aérea patentándola y proponiéndola a la fuerza
aérea inglesa, idea que fue planteada al mismo tiempo por el ingeniero alemán Hans von Ohain
provocando que durante la segunda Guerra mundial una frenética carrera entre ambos bandos
por el desarrollo de los primeros motores a reacción, tras esto la turbina
basada en el ciclo de Brayton domino como el sistema propulsivo de aeronaves.
En 1988 cuando el ingeniero rumano
Adrian Bejan realiza un estudio sobre el ciclo de Brayton ideal, demostrando 
que la eficiencia bajo la restricción de mínima generación de entropía 
corresponde a la eficiencia de Curzon-Ahlborn,
En la época actual se realizan
investigaciones y desarrollos en el ciclo Brayton, incluyendo la optimización del 
desempeño y la eficiencia en diversos aspectos, como la generación de energía y 
la propulsión.
\begin{thebibliography}{3}

\bibitem{brayton}
Wikipedia. \emph{Ciclo Brayton}. Consultado el 3 de agosto de 2025. Disponible en: \url{https://es.wikipedia.org/wiki/Ciclo_Brayton}

\bibitem{whittle}
Wikipedia. \emph{Frank Whittle}. Consultado el 3 de agosto de 2025. Disponible en: \url{https://es.wikipedia.org/wiki/Frank_Whittle}

\bibitem{bejan}
Wikipedia. \emph{Adrian Bejan}. Consultado el 3 de agosto de 2025. Disponible en: \url{https://en.wikipedia.org/wiki/Adrian_Bejan}

\end{thebibliography}
\end{document}
