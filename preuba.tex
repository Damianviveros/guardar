\documentclass[12pt]{article}
\usepackage[hidelinks]{hyperref}
\usepackage{graphicx}
\usepackage{fontenc}
\usepackage{helvet}
\usepackage[spanish]{babel}
\usepackage[utf8]{inputenc}
\usepackage{setspace}
\title{\textsf{Espacio-Tiempo cuantico\\
una cronica sobre la unificacion de las fuerzas}}
\date{junio del 2023}
\author{Damian Viveros Alvarez}
\begin{document}
\begin{figure}
    \includegraphics[width=60mm]{uam.png}
    \raggedleft  
\end{figure}
\maketitle{}
\section*{Introduccion}
\spacing{1.5}
\paragraph*{}
\textsf{"Teoria del todo. Los fisicos describen con este término simple la posibilidad de encontrar algun dia una unica y gran ecuacion 
que les permita explicar todos los fenomenos de la naturaleza'', así nos introduce el autor a una historia la cual nos lleva por grandes 
descubrimiento de genios, los cuales pensaron en su momento que toda la naturaleza podía resumirse en un conjunto único de reglas o "leyes'',
pero antes de empezar quisiera desglosar punto por punto los objetivos de este ensayo.\\ Primero este ensayo no será una crítica
hacia el libro, si no una pequeña reflexión a partir de mi punto de vista sobre este, segundo trataré de explicar de manera
explicita la razón de porque elegí este libro y tercero sera una pequeño recorrido por cada capitulo para explicar de que trata
el libro y como se relaciona con temas previamente vistos en clase;}
\section*{Antecedentes}
\spacing{1.5}
\paragraph*{}
\textsf{Ahora antes de continuar el tema tenemos que dar algunos antecedentes.\\ El libro fue publicado el 1 de enero del 2015
y su autor es el profesor de la universidad de granada Arturo Quirantes, el libro pertenece a la coleccion
un paseo por el cosmos de la casa productora National geografic es un libro de divulgacion cientifica tratando principalmente
el tema de las teorias del todo y la unificacion de las fuerzas.\\
Dicho esto}
\section*{Newton y Maxwell los unificadores}
\spacing{1.5}
\paragraph*{}
\textsf{Como tal, el primer capítulo del libro se llama fuerzas del universo y nos explica como surgieron cada una de las fuerzas
que actualmente usamos para describir todos los fenómenos físicos, pero lo más interesante de este capítulo a mi parecer es
el tratamiento que le da a Newton y Maxwell como los primeros grandes unificadores, el primero siendo considerado con tal título por 
unificar las leyes del cielo y la tierra, para explicarlo nos tenemos que remontar a la época de Aristóteles, en la cual esté mismo
consideraba que la tierra que era un conjunto imperfecto de seres, por lo cual no podía ser regido por las mismas leyes que regían 
algo tan puro como lo era para el reino del cielo, conforme pasaron los años se les empezó a llamar mundo sublunar (tierra) y mundo 
Supra lunar (cielo) a estos dos mundos tan diferentes, ahora explicado esto, Newton es considerado un gran unificador por el hecho de 
demostrar con sus leyes de la mecánica y su ley de gravitación universal que tanto el mundo sub y supra lunar se rigen por el mismo 
conjunto de leyes, mientras tanto Maxwell, como la mayoría de gente curtida en el tema, sabe, él fue el que unifico fenómenos tan 
distintos como eran la electricidad, el magnetismo y los fenómenos ópticos, en conjunto de leyes conocidas como las leyes del 
electromagnetismo de Maxwell o leyes de Maxwell para abreviar, como dije para mí, esto es lo más interesante del capítulo a resaltar
pero no puedo dejar de mencionar el cómo trata también el origen de las otras dos fuerzas, las cuales son la interacción nuclear fuerte y
la interacción nuclear débil, para resumir estas dos fuerzas surgen por culpa del descubrimiento de el nucleo atomico y las particulas que 
lo conforman haciendo que las fuerzas que ya teniamos establecidas fueran insuficientes para explicar dos fenomenos ocurridos dentro del nucleo,
uno era como se mantenia unido el nucleo sabiendo que la fuerza electrica repulsiva entre las particulas conocidas como protones era mas fuerte
que la fuerza de gravedad a esa escala por lo cual no podria mantenerse unido, surgiendo la necesidad de agregar una tercera fuerza, y por ultimo
la interacion debil la cual surgio a causa de la incapacidad de explicar por que ocurria la desintegracion beta dentro del nucleo, de dos fuerzas
a cuatro fuerzas que debemos unificar complicado, pero con la union de la nuclear debil con la electromagnetica en la teoria electrodebil hay una
esperanza de llegar a esa teoria del todo. este primer capitulo es interesante y expone muy bien como sera el resto del libro}
\section*{Espacio y tiempo grande y pequeño}
\paragraph*{}
\spacing{1.5}
\textsf{siguiendo con el recorrido por el libro nos introducimos al capitulo dos, es un capitulo que a mi parecer es el menos interesante
por lo cual no tengo mucho que decir sobre el mas alla que nos lleva por la historia que ha sido contada muchas veces antes sobre el origen
de las dos teorias mas importantes de la fisica moderna, de las que ya he hablando anterior mente, la cuantica descubierta por culpa de la 
de la radiacion termica, y la relatividad a causa de no saber por que medio se propaga la luz y los problemas relacionados con las leyes de 
Newton a velocidades luminicas, algo que me parece ironico es que en el fondo ambas teorias nacieron por el mismo problema y es que la ecuaciones 
del electromagnetismo de Maxwell, consideradas en su conjunto como la mas grande unificacion de los fenomenos fisicos, no eran suficientes para
explicar dos fenomenos que tenian una relacion intima con el electromagnetismo, y al mismo tiempo el origen de estas dos teorias nos hicieron 
cambiar para siempre los paradigmas de la fisica alejandonos de la unificacion de la fisica que perseguiamos hace mucho.}
\section*{Una primera unificacion}
\paragraph*{}
\spacing{1.5}
\textsf{Este capítulo como tal solo explica como antes de la relatividad general de Einstein muchos físicos y matemáticos trataron de unificar los
Problemáticos fenómenos electromagnéticos con la gravedad, entre ellos destacan David Hilbert un matemático alemán el cual fue en su tiempo
una leyenda viva y pudiendo haber descubierto las leyes de la relatividad antes que Einstein, Theodor Kaluza y Oskar Klein, los cuales trabajaron
en una teoría la cual tomaba al electromagnetismo como vibraciones de objetos de cinco dimensiones en nuestras 4 dimensiones siendo una teoría 
muy odiada por el hecho de agregar dimensiones extra de las cuales no podemos tener evidencia experimental, irónico, tomando en cuenta que actualmente
las teorías más aceptadas para tratar de ser teorías del todo utilizan hasta más de 10 dimensiones espaciales plegadas de las cuales no podemos
tener evidencia experimental}
\section*{¿Teorias incompatibles?}
\paragraph*{}
\spacing{1.5}
\textsf{El tercer capítulo se podría resumir en la misma historia de siempre, del porqué la relatividad y la cuántica son dos teorías incompatibles,
y nos presenta como Einstein tratando de negar la interpretación de Copenhague de la mecánica cuántica, y junto con Podolsky y Rosen crean la famosa paradoja EPR
y también como punto final del capítulo es la simple explicación de porque estas dos teorías son incompatibles y los puntos por los cuales
por los cuales una u otra no se pueden unificar, un capítulo muy corto, la verdad muy poco interesante, pero sirve de preludio para pasar al siguiente}
\section*{Hacia la gravedad cuantica}
\paragraph*{}
\spacing{1.5}
\textsf{Y llegamos al quinto y último capítulo, que como dije para mí es el que más me interesó por lo que expuse al 
principio ya que este habla sobre varias candidatas a teorías cuánticas de gravedad que es el objetivo al que quería 
llegar, pero antes de llegar a esto nos ponen en contexto de como, aunque parecía imposible, la relatividad y la cuántica se unen
gracias a físicos como lo son Dirac, Feynman, Politzer, Glasgow, etc. y gracias a estos genios  
se ha encontrado una forma de unificar la relatividad especial y la cuántica en las conocidas teorías cuánticas de campos como lo son 
la electrodinámica cuántica QED por sus siglas en inglés, la cromodinamica cuántica, incluso el modelo estándar 
de partículas, así que todavía hay esperanza que estás dos teorías incompatibles no lo sean tanto a nivel fundamental;\\
y así llegamos a las teorías cuánticas de gravedad que como dije las toca muy superficialmente, lo cual me pareció decepcionante, pero compensa esa poca profundidad con el 
planteamiento final de los problemas que ni aunque logremos encontrar la manera de unificar la cuántica y la 
relatividad general llegaremos a una respuesta clara y satisfactoria, incluso llegando a ninguna respuesta a demás de suponer algún escenario posible dependiendo
de sí existe o no una teoría unificadora.\\
para finalizar con este recorrido a lo largo de todo el libro capítulo por capítulo
creo que no me es necesario explicar cómo se relaciona cada capítulo o tema del libro con la física vista previamente 
en las clases, ya que este libro abarca demasiados temas, los cuales son muy claros, desde la mecánica de Newton hasta la unificación de la
relatividad especial y la cuántica por parte de Dirac.}
\section*{ Motivos}
\paragraph*{}
\spacing{1.5}
\textsf{LLegando a este punto creo que es pertinente decribir las razones que me llevaron a elegir este libro y estas razones son
unicamente dos y bastante sencillas.\\La primera es que es un libro que tendia pendiente de leer, pues vera querido lector yo tengo
varios libros de la misma coleccion a la que pertenece este libro y este era el siguiente en mi lista para leer;\\
Y la segunda razon es simplemente que el tema de la unificacion de la cuantica y la relatividad siempre ha sido mi tema favorito
de la fisica teorica y como tal he leido varios otros libros que describen este tema, pero la mayoria se enfoca solo en las teorias
candidatas a teorias del todo y no va mas alla hablando de las candidatas a teorias cuanticas de gravedad y por el titulo del libro
vino a mi pensamiento que este trataria todas esas otras teorias, y a pesar de que si las trata, lo hace superficialmente.\\
Y como se fue viendo a lo largo de todo el pequeño recorrido que hice capitulo por capitulo del libro en realidad este libro se 
podria tratar como una cronica (por eso el titulo del ensayo) de como pasamos de la esperanza de que toda la fisica se pueda
explicar por una sola teoria a llegar a la busqueda actual de unir dos teorias que describen el universo de excelente manera
pero comportandose como dos señores que no se hablan entre si, uno describiendo lo microscopico y otro lo macroscopico.\\
En general eso seria por lo que lo escogi y por lo mismo mis expectativas eran muy altas pero lamentablemente no las cumplio, y 
con eso no quiero decir que es un mal libro es execelente y explica varios temas complicados de manera sencilla y facil de entender
incluso para aquel que no sabe mucho de ciencia }
\section*{conclusion}
\paragraph*{}
\spacing{1.5}
\textsf{Y llegando al final de este ensayo quiero expresar como resumen y reflexión que talvez nunca lleguemos a esa afamada teoría del todo 
que unifique toda la física, la cual venimos persiguiendo desde tiempos de Newton y ha pasado por varias fases desde Maxwell
con la unión de los fenómenos eléctricos, magnéticos y ópticos, la creación de la cuántica, el descubrimiento de la
relatividad, los intentos de unión de ambas, las invariancias gauge, la creación de las teorías cuánticas de campos
y la creación del modelo estándar, etc., y aunque logremos llegar a esa unificación, no será el fin de la física como 
lo especulaba lord kelvin a finales del siglo XX todavía hay problemas que tenemos que resolver que ni con la unión 
de la cuántica y la relatividad general podremos resolver todos estos problemas.\\ Al final no me arrepiento de haber leído 
este libro si bien las expectativas que tenía no se cumplieron, por lo menos descubrí un excelente libro de divulgación, el cual 
si bien mi objetivo principal con este libro era ver teorías de gravedad cuántica diferentes a la teoría de cuerdas o a la gravedad
Cuántica de bucles que han sido siempre lo más visto en divulgación científica cuando hablas de este tema, me fue agradable su lectura
e interesante los temas que abordo aparte de las teorías del todo}
\newpage
\section*{Bibliografia}
\paragraph*{1}
\textsf{colaboradores de Wikipedia. (2023). Gravedad cuántica. Wikipedia, La Enciclopedia Libre.}
\url{https://es.wikipedia.org/wiki/Gravedad_cuántica}
\paragraph*{2}
\textsf{Espacio-tiempo cuántico: en busca de una teoría del todo. (2015).
}
\paragraph*{3}
\textsf{Greene, B. R. (2006). El universo elegante. Supercuerdas, dimensiones ocultas y la búsqueda de una teoría definitiva.}
\end{document}