\documentclass{article}
\usepackage[utf8]{inputenc}
\usepackage[spanish]{babel}
\begin{document}
\section*{Pre-lectura 4}
\paragraph*{}
hasta ahora cuando hemos hablado de fluidos lo hemos hecho en el caso particular donde podemos despreciar su viscosidad,
pero para empezar a hablar de la viscosidad tenemos que definir la, y esta se define como la resistencia que ofrece un fluido a 
moverse, esta normalmente es fácil de medir ya que existen aparatos para ellos que se llaman viscosimetros que usan la 
llamada ley de stokes para medir la viscosidad, ¿pero que es la ley de stokes?;
bueno esta es una ley enunciada por primera vez por el físico irlandés George Gabriel Stokes la cual se obtiene al resolver un caso particular
de las ecuaciones de Navier-Stokes las cuales son una conjunto de ecuaciones en derivadas parciales que describen el comportamiento de los fluidos
siendo estas las que describen el comportamiento de todos los fluidos sin importar el tipo de fluido, de hecho hasta ahora todas las ecuaciones 
para el movimiento de los fluidos como la ecuación de continuidad y la ecuación de Bernoulli han sido casos particulares de estas ecuaciones,
entonces en el caso de la ley de stokes se supone un fluido en regimen estacionario con un flujo laminar,
en este caso ahora dejamos caer un cuerpo esférico con densidad y volumen conocidos  dentro del fluido como sabemos que el fluido ejerce una fuerza de flotación 
pero en este caso es despreciable lo que nos importa es el momento en que el objeto alcanza su velocidad terminal que es la velocidad maxima a la cual 
el objeto podrá ir ya que aunque este bajo la aceleración de la gravedad este ya no podrá acelerar mas allá de esta velocidad, esto viene determinado principalmente
por las condiciones del fluido pero en el caso especifico de la ley de stokes este se trata de un fluido en regimen estacionario, con un flujo laminar
determinado por un numero de Reynolds bajo, siendo el numero de Reynolds un numero que determina que tipo de flujo es el que se estudia siendo que cuando este 
es alto el flujo se trata de un flujo turbulento y cuando es bajo se trata de un flujo laminar, ahora pensando esto como un llenar un tubo de un fluido,
y después ponerle una esfera de tamaño y densidad conocidas la distancia a la que esta esfera alcance su velocidad terminal sera solamente
determinada por la viscosidad del fluido asi traduciendo la distancia a la que la esfera alcanza esta velocidad a su viscosidad,
esta ley a sido probada muchas veces con varios fluidos y de ahi su importancia para averiguar la viscosidad real de un fluido, teniendo
varias aplicaciones en la ingeniería incluso desempeñando un papel critico en la investigación de al menos 3 premios nobel.
\end{document}