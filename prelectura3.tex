\documentclass{article}
\usepackage[utf8]{inputenc}
\usepackage[spanish]{babel} 
\begin{document}
\section*{Medidas de presión}
\paragraph*{}
Cuando hablamos de presión no nos referimos a una única presión sino que esta se divide en varias
la presión barométrica, presión absoluta y la presión manométrica, la primera tiene que ver con la presión
de la atmósfera cuya variación es muy pequeña respecto a una variación estándar, esta presión se mide con un 
barómetro el cual fue inventado en 1643 por Evangelista Torricelli un físico italiano, el instrumento consistía 
en llenar un tubo de cristal con mercurio y voltearlo en otro recipiente lleno con mercurio, según el principio de 
pascal como la presión se reparte de manera uniforme en todos los lados del fluido entonces la presión del fluido
inferior evitaría que este cayera por la acción de la gravedad, pero como la atmósfera también ejerce una presión
sobre el mercurio, y como nos dice la ley fundamental de la estática de fluidos la presión de un fluido
varia con la altura, entonces al medir esta altura y saber la presión que ejerce el mercurio, la cual depende
de su densidad y la altura, se puede obtener la medida de presión que ejerce la atmósfera siendo esta
la distancia de la altura de la columna de mercurio en milímetros, suponiendo una temperatura de 0°C,
siendo esta medida conocida como milímetros de mercurio $mmHg$ , ya después se instauro el sistema métrico 
instaurando la unidad internacional 
para medir la presión los pascales, aunque normalmente se utiliza atmósferas cuando se habla de la
presión atmosférica o barométrica siendo la presión atmosférica igual a 1atm que equivale a 101,325 pa
\end{document}