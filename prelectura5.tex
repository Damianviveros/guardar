\documentclass{article}
\usepackage[spanish]{babel}
\usepackage[utf8]{inputenc}
\begin{document}
\section*{Temperatura y teoría molecular de la materia}
Normalmente nuestra percepción de temperatura esta dada por nuestros sentidos, principalmente el sentido del tacto, pero resulta que los materiales
presentan otras formas de expresar su temperatura, una de las formas en que lo hacen es comprimiéndose y expandiéndose dependiendo su temperatura,
lo cual se llama dilatación térmica, aunque esto se ve principalmente en sólidos y líquidos, en cambio los gases presentan otras formas como lo son 
un aumento o disminución de volumen o en caso de ser el volumen contante un aumento o disminución de presión, estas propiedades que suelen variar con la 
temperatura se les conoce como propiedades termométricas, y se pueden notar por ejemplo poniendo una barra de metal fría en contacto con una barra caliente,
se notara que estas barras al transferir calor entre ellas una de las barras tiende a comprimirse (la barra caliente al enfriarse) y la otra a expandirse (la barras
fría al calentarse), pero ¿porque sucede esto?, pues como sabemos la materia esta hecha de moléculas y las moléculas a su vez están hechas de átomos,
y resulta que estos átomos y moléculas se mueven alrededor del espacio, al hacer esto significa por significa que tienen energía, pero como normalmente 
las moléculas están confinadas gracias a fuerzas tales como la gravedad o fuerzas electromagnéticas entre ellas, resulta que estas chocan con otras moléculas
de su alrededor transfiriendo parte de su energía a estas y a su vez estas chocan con otras transfiriendo esa energía interna del sistema de una molécula a 
molécula pudiendo transferir esta energía interna del sistema a otro sistema que este en contacto con el primero, tal como pasa con la temperatura cuando tienes
dos sistemas en contacto y por deducción se puede decir que la temperatura en realidad es la energía del movimiento de una molécula, dándonos asi una
explicación a otro fenómeno extraño de la temperatura y es que resulta que a diferencia de otras cantidades físicas como la distancia o la masa, la temperatura no es
aditiva esto quiere decir que si tu sumas 2kg a 6kg el resultado es igual a 8kg, pero si tu pones una barra de metal a 40°C y otra a 10°C en contacto,
el sistema no tendrá 50°C si no que tendrá una temperatura de 35°C aproximadamente, esto con el modelo de las moléculas chocando se puede
entender mas fácil ya que solo se tiene que imaginar moléculas a alta velocidad chocando con moléculas a baja velocidad haciendo que al final
todas tengan una misma velocidad la cual por conservación de momento se sabe que tendrá que ser menor a la velocidad original de 
la molécula mas veloz, también explicando porque siempre que pones un objeto frio y uno caliente en contacto la temperatura fluye del objeto 
caliente al objeto frio y no viceversa, todo esto que acabo de explicar se conoce como modelo cinético de la materia una teoría que remonta
desde el 50 a.C donde Lucrecio un filosofo griego planteo que los objetos macroscópicos estáticos estaban hechos de átomos que se movían rápidamente
y chocaban entre si y aunque posteriormente físicos y matemáticos trabajaron con ideas parecidas como lo fueron Bernoulli y Clausius, no fue hasta que
James Clerk Maxwell y Ludwig Boltzmann introdujeran la mecánica analítica y posteriormente fuera comprobada por el movimiento Browniano, 
que el modelo molecular de la temperatura fuera ampliamente aceptado y tomado como una realidad siendo la base de la actual termodinámica que 
existía mucho antes que el modelo molecular de la temperatura fuera aceptado y que originalmente negaba la posibilidad de que este fuera valido.

\end{document}